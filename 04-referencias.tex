\documentclass[10pt,a4paper,titlepage]{article} %Formato del documento
\usepackage[utf8]{inputenc} %Especificamos que queremos usar utf-8 (ñ o tildes)
%Si no importamos este paquete, las ñ no aparecerían por ejemplo
\usepackage[spanish]{babel}
\usepackage{geometry} %Márgenes del documento
\geometry{top=2.54cm, left=3.1cm, right=3.1cm, bottom=2.54cm}

\usepackage{hyperref} % Permite que los títulos del índice sean enlaces directos a los apartados. Además permite añadir enlaces a internet, referencias...
\usepackage{nameref} % referenciar nombre de titulos
%%%%%%%%%%%%%%
\hypersetup{
    colorlinks,
    citecolor=black,
    filecolor=black,
    linkcolor=black,
    urlcolor=black
}
%No es realmente necesario este código, pero hace que sean más bonitos los enlaces
%%%%%%%%%%%%%%%%


%Definimos el título del documento
\title{ \textbf{ \Huge{Latex $>$ Word}} \\ Jornadas Técnicas XXVIII}
\author{
		\begin{tabular}{l}
			\multicolumn{1}{l}{GUL} \\ \hline \\
			Daniel Alejandro Rodríguez López \\
		\end{tabular}
}
%
\begin{document}

\maketitle
\newpage

%Para que el índice se incluya correctamente debemos compilarlo dos veces. En la 1º se genera y en la 2º se incluye en el documento.
\tableofcontents
\newpage

	\section{Tabla aqui Tabla allá}
	\label{intro}
	Con esto creamos una tabla de 3 columnas que tienen el texto centrado. Los estilos más comunes son \textbf{l} (letra l, no número 1) (justificado a la izquierda), \textbf{c} (centrado) y \textbf{r} (justificado a la derecha). 

		\subsection{Tabla 1}
		\begin{center}
		\begin{tabular}{l c r}
		Izquierda & Centro & Derecha \\ \hline
		Patatas & Mesa & silla \\
		Cosa1 & Cosa2 & Cosa3 \\ \hline
		\end{tabular}
		\end{center}


		Se puede jugar mucho con las tablas, como por ejemplo, combinar celdas o tablas dentro de tablas (algo complicado de hacer la 1º vez, pero luego es copiar y pegar xD)

		\newpage
		\subsection{Tabla 2}
		\begin{center}
		\begin{tabular}{l c r}
		\multicolumn{3}{c}{He juntado las 3 columnas} \\ \hline
		Izquierda & Centro & Derecha \\ \hline
		Patatas & Mesa & silla \\
		Cosa1 & Cosa2 & Cosa3 \\ \hline
		\end{tabular}
		\end{center}

	Las columnas se separan mediante \textbf{\&}. El salto de línea se realiza mediante \textbf{$\backslash$$\backslash$} y la linea negra en el salto de línea se realiza con el comando \textbf{\texttt{hline}}.

	\newpage
	\section{Otro título para poder usar referencias}
	Cosicas asdhsaidhiada patata asdadsad bambisito xDDDDDDDDDDD report \\
	Como se puede observar, si clicamos al siguiente texto \textbf{\nameref{intro}} vamos a la primera sección del documento.

\end{document}