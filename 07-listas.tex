\documentclass[10pt,a4paper,titlepage]{article} %Formato del documento
\usepackage[utf8]{inputenc} %Especificamos que queremos usar utf-8 (ñ o tildes)
%Si no importamos este paquete, las ñ no aparecerían por ejemplo
\usepackage[spanish]{babel}
\usepackage{hyperref} % Permite que los títulos del índice sean enlaces directos a los apartados. Además permite añadir enlaces a internet, referencias...
\usepackage{ulem} %para tachar
%%%%%%%%%%%%%%
\hypersetup{
    colorlinks,
    citecolor=black,
    filecolor=black,
    linkcolor=black,
    urlcolor=black
}
%No es realmente necesaria, pero hace que sean más bonitos los enlaces
%%%%%%%%%%%%%%%%
%
%Definimos el título del documento
\title{ \textbf{ \Huge{Latex $>$ Word}} \\ Jornadas Técnicas XXVIII}
\author{
		\begin{tabular}{l}
			\multicolumn{1}{l}{GUL} \\ \hline \\
			Daniel Alejandro Rodríguez López \\
		\end{tabular}
}
%
\begin{document}

\maketitle

\newpage

\section*{Listas \sout{de la compra}}
	Usar listas es bastante sencillo. Tenemos dos tipos principales de listas: Numeradas (enumerate) y No Numeradas (itemize).

	\subsection*{Tipo 1}
		\begin{enumerate}
			\item Elemento 1
			\item Elemento 2
		\end{enumerate}

	\subsection*{Tipo 2}
		\begin{itemize}
			\item Elemento 1
			\item Elemento 2
		\end{itemize}

	Las listas se crean mediante el comando $\backslash$begin{enumerate ó itemize} <cosas de tu lista> $\backslash$end{enumerate ó itemize}.

	Los elementos de la lista se agregan mediante el comando $\backslash$item.

\section*{Lista de listas}
	Tambien es posible crear subniveles dentro de las listas. Donde queramos que aparezca el nuevo subnivel escribimos $\backslash$item seguido de la lista que queremos crear.
	\begin{enumerate}
		\item \begin{enumerate}
			\item Elemento 1.1
			\item Elemento 1.2
		\end{enumerate}
		\item Elemento 2
	\end{enumerate}
\end{document}