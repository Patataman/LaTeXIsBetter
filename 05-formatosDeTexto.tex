\documentclass[10pt,a4paper,titlepage]{article} %Formato del documento
\usepackage[utf8]{inputenc} %Especificamos que queremos usar utf-8 (ñ o tildes)
%Si no importamos este paquete, las ñ no aparecerían por ejemplo
\usepackage[spanish]{babel}
\usepackage{hyperref} % Permite que los títulos del índice sean enlaces directos a los apartados. Además permite añadir enlaces a internet, referencias...
\usepackage{color} %Para cambiar el color de las letras
%%%%%%%%%%%%%%
\hypersetup{
    colorlinks,
    citecolor=black,
    filecolor=black,
    linkcolor=black,
    urlcolor=black
}
%No es realmente necesaria, pero hace que sean más bonitos los enlaces
%%%%%%%%%%%%%%%%
%
%Definimos el título del documento
\title{ \textbf{ \Huge{Latex $>$ Word}} \\ Jornadas Técnicas XVIII}
\author{
		\begin{tabular}{l}
			\multicolumn{1}{l}{GUL} \\ \hline \\
			Daniel Alejandro Rodríguez López \\
		\end{tabular}
}
%
\begin{document}

\maketitle

\newpage

\section*{Poner bonito el documento}
	Aunque a estas alturas ya se ha usado en otros ejemplos, vamos a realizar una breve descripción de los formatos de texto más utilizados:
	\begin{enumerate}
		\item \textbf{Negrita:} La letras en negrita se generan mediante \textbf{$\backslash$textbf\{$<$Texto que queremos en negrita$>$\}} = \textbf{Ola k ase}.
		\item \textbf{Cursiva:} La letras en cursiva se generan mediante \textbf{$\backslash$textit\{$<$Texto que queremos en cursiva$>$\}} = \textit{Ola k ase}.
		\item \textbf{Subrayado:} La letras subrayadas se generan mediante \textbf{$\backslash$underline\{$<$Texto que queremos subrayado$>$\}} = \underline{Ola k ase}.
		\item \textbf{Estilo código:} La letras estilo código se generan mediante \textbf{$\backslash$texttt\{$<$Texto que queremos en código$>$\}} = \texttt{printf(`Ola k ase');}.
		\item \textbf{Cambiar color:} Necesitaremos importar el paquete \textbf{color} y se usa mediante \textbf{$\backslash$color\{$<$Color que queremos$>$\}\{$<$Texto que queremos coloreado$>$\}} = \color{blue}{Color azul}.
	\end{enumerate}

\end{document}