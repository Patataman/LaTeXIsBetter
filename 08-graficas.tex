\documentclass[10pt,a4paper,titlepage]{article} %Formato del documento
\usepackage[utf8]{inputenc} %Especificamos que queremos usar utf-8 (ñ o tildes)
%Si no importamos este paquete, las ñ no aparecerían por ejemplo
\usepackage[spanish]{babel}
\usepackage{graphicx}
\usepackage{pgfplots}
\usepackage{hyperref} % Permite que los títulos del índice sean enlaces directos a los apartados. Además permite añadir enlaces a internet, referencias...
\usepackage{float}
%%%%%%%%%%%%%%
\hypersetup{
    colorlinks,
    citecolor=black,
    filecolor=black,
    linkcolor=black,
    urlcolor=black
}
%No es realmente necesaria, pero hace que sean más bonitos los enlaces
%%%%%%%%%%%%%%%%
%
%Definimos el título del documento
\title{ \textbf{ \Huge{Latex $>$ Word}} \\ Jornadas Técnicas XXVIII}
\author{
		\begin{tabular}{l}
			\multicolumn{1}{l}{GUL} \\ \hline \\
			Daniel Alejandro Rodríguez López \\
		\end{tabular}
}
%
\begin{document}

\maketitle

\newpage

\section*{Gráficas}

	\begin{figure}[H]
		\centering
		\begin{tikzpicture}
			\begin{axis}[
			 	title={Grafica bonica},
				ybar,
				symbolic x coords={Columna1, Columna2},
				xtick=data,
				ylabel={Cosas del Y},
				%bar width=10, anchura de las barras, por defecto 10.
				ymax=50, %valor máximo del eje Y
				enlargelimits=0.07,
				legend style={at={(0.5,-0.15)}, anchor=north,legend columns=-1},
				ymajorgrids = true,
				enlarge x limits={abs=1cm}, %margen a los extremos de la gráfica
				%x tick label style={rotate=40, anchor=east, align=right,text width=1.5cm}, %estilo de las etiquetas del eje X
				nodes near coords,
				%width = 17cm, anchura de la grafica
				%height = 8cm altura de la gráfica
			]
			%cada addplot son conjuntos de columnas
			\addplot coordinates { (Columna1,10) (Columna2,5) };
			\addplot coordinates { (Columna1,23) (Columna2,18) };
			\addplot coordinates { (Columna1,15) (Columna2,20) };

			\legend{Categoria 1, Categoria 2, Categoria 3}
			\end{axis}
		\end{tikzpicture}
		\caption{Nombre para la gráfica}
		\end{figure}
\end{document}